%%%%%%%%%%%%%%%%%%%%%%%%%%%%%%%%%%%%%%%%%
% Medium Length Professional CV
% LaTeX Template
% Version 2.0 (8/5/13)
%
% This template has been downloaded from:
% http://www.LaTeXTemplates.com
%
% Original author:
% Trey Hunner (http://www.treyhunner.com/)
%
% Important note:
% This template requires the resume.cls file to be in the same directory as the
% .tex file. The resume.cls file provides the resume style used for structuring the
% document.
%
%%%%%%%%%%%%%%%%%%%%%%%%%%%%%%%%%%%%%%%%%

%----------------------------------------------------------------------------------------
%	PACKAGES AND OTHER DOCUMENT CONFIGURATIONS
%----------------------------------------------------------------------------------------

\documentclass{resume} % Use the custom resume.cls style

% Finally, give us PDF bookmarks
\usepackage{color,hyperref}
\definecolor{darkblue}{rgb}{0.0,0.0,0.3}
\hypersetup{colorlinks,breaklinks,
            linkcolor=blue,urlcolor=blue,
            anchorcolor=blue,citecolor=blue}
            
% Uses hyperref to link DOI
\newcommand\doilink[1]{\href{http://dx.doi.org/#1}{#1}}
\newcommand\doi[1]{doi:\doilink{#1}}

% For \url{SOME_URL}, links SOME_URL to the url SOME_URL
%\providecommand*\url[1]{\href{#1}{#1}}
% Same as above, but pretty-prints SOME_URL in teletype fixed-width font
\renewcommand*\url[1]{\href{#1}{\texttt{#1}}}

\usepackage[left=1in,top=1in,right=1in,bottom=1in]{geometry} % Document margins
\newcommand{\tab}[1]{\hspace{.15\textwidth}\rlap{#1}} 
\newcommand{\itab}[1]{\hspace{0em}\rlap{#1}}
\name{\huge OANA VESA} % Your name
\address{Sunnyvale, California} 
%\address{P.O. Box 30001, MSC 4500, Las Cruces, NM 88003-8001, USA}
% \address{Sunnyvale, California 94089} 
%\address{806A Mormon Pl, Las Cruces, NM 88011} % Your address
%\address{123 Pleasant Lane \\ City, State 12345} % Your secondary address (optional)
\address{586-344-6809 \\ ovesa@nmsu.edu \\  \href{https://github.com/ovesa}{\underline{github.com/ovesa}}}  % Your phone number and email

%\address{Pronunciation: WAH-nah VEH-sa - Other language: Romanian}
\begin{document}
%----------------------------------------------------------------------------------------
%	EDUCATION SECTION
%----------------------------------------------------------------------------------------
\vspace{-0.1in}
\begin{rSection}{\textbf{Education}}
\vspace{0.02in}

{\bf Ph.D. Candidate in Astronomy, New Mexico State University} \hfill {Expected 2024}
\vspace{0.04in}
\\ 
{\bf Working Thesis Title:} 
\emph{Harnessing the Untapped Potential of Atmospheric Gravity Waves and Chromospheric Swirls to Map Out the Solar Atmosphere}
% \emph{Characterization of atmospheric gravity waves and magnetic tornadoes as tools to probe the solar atmosphere}
\vspace{0.04in}
\\ 
{\bf Thesis Committee:} Juie Shetye, Jason Jackiewicz, Jon Holtzman, Laura Boucheron
\vspace{0.15in}

{\bf B.A. in Physics \& Mathematics with Honors, Albion College}  \hfill {05/2018}
% \hfill {08/2014 -- 05/2018}
\vspace{0.04in} \\
{\emph{Prentiss M. Brown Honors Program} }
\vspace{0.04in} \\
{\bf Thesis Title:} \emph{Analysis of the Gaia RVS Region in ESPaDOnS Spectra of Asteroseismic \\ Calibration Stars}
\vspace{0.04in}
\\ 
{\bf Advisor:} Nicolle Zellner 
\end{rSection}

%----------------------------------------------------------------------------------------
%	RESEARCH INTEREST
%----------------------------------------------------------------------------------------
\vspace{-0.1in}
\begin{rSection}{\textbf{Research Interests}}
\vspace{0.02in}
High-resolution, multi-height observations from multiple viewing angles to characterize \\ oscillatory and vortex flows in the lower solar atmosphere.
\vspace{0.02in}
\begin{itemize}
    \item Atmospheric Gravity Waves -- Characterization of their behavior and potential as \\ atmospheric diagnostics -- Theory and observations
    \item Small-scale Vortex Flows (Chromospheric Swirls) -- 
    Characterization of their formation, evolution, and role in transferring energy and mass -- Theory and observations
\end{itemize}
\end{rSection}


%----------------------------------------------------------------------------------------
%	PAPERS
%----------------------------------------------------------------------------------------

\vspace{-0.1in}
\begin{rSection}{\textbf{Refereed Publications}}
\vspace{0.02in}
\href{https://iopscience.iop.org/article/10.3847/1538-4357/acd930}{Multiheight Observations of Atmospheric Gravity Waves at Solar Disk Center} \\ {\bf Vesa, O.}, Jackiewicz, J., and Readorn, K., The Astrophysical Journal, Volume 952, Issue 1, article id. 58, 18 pp. (07/2023)
\end{rSection} 

%----------------------------------------------------------------------------------------
%	AWARDS SECTION
%----------------------------------------------------------------------------------------

\vspace{-0.1in}
\begin{rSection}{\textbf{AWARDS, HONORS and GRANTS}}
\vspace{0.02in}

{\href{https://astro.nmsu.edu/fellowships/zia.html}{Zia Award}}
\hfill{2023} \\
\emph{``...recognizes outstanding research by a graduate student in the NMSU Astronomy \\ Department.''}


{\href{https://astro.nmsu.edu/fellowships/rappaport.html}{The Dr. Barry Neil Rappaport Endowed Memorial Scholarship}}
\hfill{2023} \\
\emph{``...recognition of an exceptional record of public outreach and service or for an \\ exceptional completed research project in observational astronomy which demonstrates \\ excellence and breadth.''}

{Co-I on Nationwide Eclipse Ballooning Project (NEBP) for New Mexico State \hfill{2022} \\ University} \\
\emph{PI: Juie Shetye;} \emph{Atmospheric Science Track Team}

{\href{https://astro.nmsu.edu/fellowships/murrell.html}{A. Scott Murrell Memorial Endowed Scholarship Fund}}
\hfill{2022} \\
\emph{``...Recognizes outstanding research or professional development, and related \\ accomplishments that raise the visibility of the NMSU Astronomy Department''}

{New Mexico Space Grant Consortium Graduate Research Fellowship } \hfill{2021, 2022} \\
\emph{``Harnessing the Untapped Potential of the Solar Tornadoes''; awarded \$10,000}
%\end{itemize}
\end{rSection}



%----------------------------------------------------------------------------------------
%	In the news
%----------------------------------------------------------------------------------------
\vspace{-0.1in}
\begin{rSection}{\textbf{IN THE NEWS}}
\vspace{0.02in}

% {242\textsuperscript{nd} AAS Press Release. \href{https://aas.org/sites/default/files/2023-06/AAS242_Mon2_OanaVesa.pdf}{Characterizing Tornadoes on the Sun} \hfill{06/05/2023}} 

{Santa Fe New Mexican Article. \href{https://www.santafenewmexican.com/news/local_news/nmsu-researchers-shine-light-on-solar-tornadoes/article_e44fa558-096b-11ee-97c9-132701cfe264.html}{NMSU Researchers Shine Light on Solar Tornadoes}} \hfill{06/17/2023} 

{NMSU Press Release. \href{https://newsroom.nmsu.edu/news/nmsu-researchers-study-solar-tornadoes--impact--news-conference-in-albuquerque-june-5/s/0bf4893e-8456-4c9c-b68e-1a4f1f88cf5b}{NMSU Researchers Study Solar Tornadoes’ Impact, News}}\hfill{06/01/2023} \\
{\href{https://newsroom.nmsu.edu/news/nmsu-researchers-study-solar-tornadoes--impact--news-conference-in-albuquerque-june-5/s/0bf4893e-8456-4c9c-b68e-1a4f1f88cf5b}{Conference in Albuquerque June 5}} 

{Las Cruces Sun News Article. \href{https://www.lcsun-news.com/story/life/2022/12/05/star-news-nmsu-team-to-use-hot-air-balloons-to-study-sun-amid-eclipses/69699732007/}{NMSU Team to Use Hot-air Balloons to Study} \hfill{12/05/2022}} \\
\href{https://www.lcsun-news.com/story/life/2022/12/05/star-news-nmsu-team-to-use-hot-air-balloons-to-study-sun-amid-eclipses/69699732007/}{Solar Effects Amid Eclipses}  
\end{rSection} 


%----------------------------------------------------------------------------------------
%	PRESENTATIONS
%----------------------------------------------------------------------------------------
\vspace{-0.1in}
\begin{rSection}{\textbf{Invited and Press Talks}}
\vspace{0.02in}

{Stanford Solar Seminar \hfill{01/2024} \\
Title: \emph{``Multi-Height Observations of Propagating Atmospheric Gravity Waves''}}

{Press Talk for the 242\textsuperscript{nd} American Astronomical Society Meeting in \hfill{06/2023} \\ Albuquerque, New Mexico} \\
Title: \href{https://aas.org/sites/default/files/2023-06/AAS242_Mon2_OanaVesa.pdf}{\emph{Characterizing Tornadoes on the Sun}}

{Albion College Mathematics \& Computer Science Department Colloquium Series \hfill{04/2021} \\
Title: \emph{``Atmospheric Gravity Waves in the Magnetized Solar Atmosphere''}}

{Preparing for DKIST: Image Processing and Time Series Workshop held at \hfill{01/2020} \\  California State University, Northridge  \\
Title: \emph{``Gravity Waves in the Photosphere''}}
\end{rSection}

\vspace{-0.1in}
\begin{rSection}{\textbf{Conference Presentations}}
\vspace{0.02in}

{Poster Presentation. The American Geophysical Union (AGU) Fall Meeting 2023 \hfill{12/2023} \\  Title: \emph{``Multi-Height Observations of Propagating Atmospheric Gravity Waves''}}

{Podium Talk. 54\textsuperscript{th} Solar Physics Division Meeting \hfill{08/2023} \\ Title: \emph{``Unlocking the Secrets of Atmospheric Gravity Waves on the Quiet Sun: Observational Insights''}}

{Poster Presentation. 54\textsuperscript{th} Solar Physics Division Meeting \hfill{08/2023} \\  Title: \emph{``Characterization of Chromospheric Swirls on the Quiet Sun''}}

{IPoster Presentation. 242\textsuperscript{nd} American Astronomical Society  Meeting \hfill{06/2023} \\ Title: \emph{``Characterization of Chromospheric Swirls on the Quiet Sun''}}

{Virtual Talk. Joint Scientific Assembly IAGA-IASPEI \hfill{08/2021} \\ Title: \emph{``Atmospheric Gravity Waves in the Magnetized Lower Solar Atmosphere''}} 

{Virtual Talk. 36\textsuperscript{th} Annual New Mexico Symposium \hfill{11/2020} \\ Title: \emph{``The Propagation of Atmospheric Gravity Waves in the Magnetic Solar Atmosphere''}}
%\vspace{0.1in}

{IPoster Presentation. 51\textsuperscript{st} Solar Physics Division Meeting \hfill{08/2020} \\ Title: \emph{``Atmospheric Gravity Waves in the Magnetized Solar Atmosphere''}}

% {Informal Talk. NMSU Pizza Lunch \hfill{12/2019} \\ Title: \emph{``IBIS' Last Remarks: Gravity Waves''}}

{Podium Talk. 29\textsuperscript{th} Annual Elkin R. Isaac Student Research Symposium \hfill{04/2018} \\ Title: \emph{``Analysis of the Gaia RVS Region in ESPaDOnS Spectra of Asteroseismic Calibration Stars''}}

{Poster Presentation. 231\textsuperscript{st} American Astronomical Society Meeting \hfill{01/2018} \\ Title: \emph{``Analysis of the Gaia RVS Region in ESPaDOnS Spectra of Asteroseismic Calibration
Stars''}} 

{Poster Presentation. 229\textsuperscript{th} American Astronomical Society Meeting \hfill{01/2017} \\ Title: \emph{```The Evolution of Starspots on LO Pegasi''}}
\end{rSection}

%------------------------------------------------------------------
%	JOBS SECTION
%----------------------------------------------------------------------------------------

\vspace{-0.1in}
\begin{rSection}{\textbf{Research Experience}}
\vspace{0.02in}

{NSF REU Intern, {University of Hawai'i-Manoa}} \hfill {Summer 2017}
\vspace{0.04in}
\\
{Topic:} \emph{Analyzing the Gaia RVS Region in ESPaDOnS Spectra}
\vspace{0.04in}
\\
{Advisors:} Daniel Huber, Eric Gaidos
\vspace{0.05in}

{NSF REU Intern, Ohio Wesleyan University} \hfill {Summer 2016}
\vspace{0.04in}
\\
{ Topic:} \emph{Analysis of starspots on the young solar analog LO Pegasi}
\vspace{0.04in}
\\
{ Advisor:} Robert Harmon
\vspace{0.05in}

{ Summer Research Assistant, Albion College} \hfill {Summer 2015}
\vspace{0.04in}
\\
{ Topic:} \emph{Analysis of the chemical composition and ages of lunar impact \\ glass samples from Apollo 14, 16, and 17 sites}
\vspace{0.04in}
\\
{ Advisor:}  Nicolle Zellner
\vspace{0.05in}
\end{rSection}
%\vspace{0.2in}


%----------------------------------------------------------------------------------------
%	TEACHING EXPERIENCE
%----------------------------------------------------------------------------------------

\vspace{-0.1in}
\begin{rSection}{\textbf{TEACHING and Mentoring EXPERIENCE}}
\vspace{0.02in}

{Co-Instructor with Juie Shetye for ASTR 400: Undergrad Research} \hfill{03/2023} \\
{\emph{Developed course material and lectured for three classes about Earth-based atmospheric \\ gravity waves and their connection to solar eclipses}}

% {Developed and lectured three classes on Earth's atmospheric gravity waves \hfill{03/2023} \\ for ASTR 400: Undergrad Research}

{Graduate Student Mentor for the Astronomy Department’s \hfill{08/2021 -- 08/2022} \\ Undergraduate Mentoring Program for Astronomy Minors} 

{Graduate Teaching Assistant for ASTR 110: Introduction to Astronomy} \hfill{08/2018 -- 05/2019}

{Undergraduate Teaching Assistant for PHYS 245: Electronics} \hfill{08/2016 -- 12/2016}

{Albion College Mathematics Tutor}  \hfill{ 01/2016 -- 05/2018}

{Albion College Physics Peer Mentor}  \hfill{08/2015 -- 05/2018}
\end{rSection}


%----------------------------------------------------------------------------------------
%	LEADERSHP
%----------------------------------------------------------------------------------------

\vspace{-0.1in}
\begin{rSection}{\textbf{LEADERSHIP and SERVICE}}
\vspace{0.02in}

{Vice-President of the NMSU Astronomy Graduate Student Organization \hfill{08/2020 -- 08/2023}} 

{Graduate Student Outreach Coordinator for the Astronomy Department} \hfill{08/2020 -- 08/2023}

{\href{https://prescientist.org/}{Letters to a Pre-Scientist (LPS)} Pen Pal Volunteer} \hfill{08/2020 -- 08/2023}
\end{rSection} 


%----------------------------------------------------------------------------------------
%	OBSERVING EXPERIENCE SECTION
%----------------------------------------------------------------------------------------

\vspace{-0.1in}
\begin{rSection}{\textbf{Observing Experience}}
\vspace{0.02in}

{Dunn Solar Telescope (ROSA, FIRS)} \hfill{2022}

{Dunn Solar Telescope (IBIS, ROSA)} \hfill{2019}

%\vspace{-0.05in}
\end{rSection} 
\newpage



\vspace{-0.1in}
\begin{rSection}{\textbf{Workshops and Summer Schools}}
\vspace{0.02in}

{Preparing for DKIST: He I Diagnostics in the Solar Atmosphere Workshop \hfill{02/2022}} 

{Preparing for DKIST: An Introduction to Chromospheric Diagnostics Workshop} \hfill{07/2021}

{Pennsylvania State University's Center for Astrostatistics: Summer School \hfill{06/2021} \\ in Statistics for Astronomers XVI}

{Preparing for DKIST: Milne-Eddington Spectro-polarimetric Inversions Workshop \hfill{07/2020}}

{Preparing for DKIST: Image Processing and Time Series Workshop\hfill{01/2020}}

{Preparing for DKIST: An Introduction to Ground-based Data  Workshop \hfill{06/2019}}

{DKIST Critical Science Plan Workshop on 
Wave Generation and Propagation \hfill{12/2018}}
\end{rSection} 


%----------------------------------------------------------------------------------------
%	SKILLS AND INTERESTS SECTION
%----------------------------------------------------------------------------------------

\vspace{-0.1in}
\begin{rSection}{\textbf{Relevant Skills}}
\vspace{0.02in}

\begin{tabular}{@{} >{\bfseries}l@{\hspace{6ex}}l}
Programming & Proficient in Python and IDL; \\ &  Some experience in Fortran, MATLAB, R, and Perl
\vspace{0.12in} 
\\
Data Reduction & Substantial experience in narrowband and broadband data \\ & reduction for ground-based instruments
\vspace{0.12in}
\\
Languages & English (native), Romanian (native), Spanish (beginner)

%Extracurricular & Rock climbing (\href{https://www.mountainproject.com/user/110312369/drew-chojnowski}{Mountain Project} administrator, 2017--present), hiking, \\
   %             & kayaking, catch \& release fishing \\
\end{tabular}
\end{rSection}


\clearpage



\end{document}